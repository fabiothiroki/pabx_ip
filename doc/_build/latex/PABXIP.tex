% Generated by Sphinx.
\def\sphinxdocclass{report}
\documentclass[letterpaper,10pt,brazil]{sphinxmanual}
\usepackage[utf8]{inputenc}
\DeclareUnicodeCharacter{00A0}{\nobreakspace}
\usepackage[T1]{fontenc}
\usepackage{babel}
\usepackage{times}
\usepackage[Sonny]{fncychap}
\usepackage{longtable}
\usepackage{sphinx}
\usepackage{multirow}


\title{PABX IP Documentation}
\date{13/03/2012}
\release{0.1}
\author{Fabio Hiroki}
\newcommand{\sphinxlogo}{}
\renewcommand{\releasename}{Versão}
\makeindex

\makeatletter
\def\PYG@reset{\let\PYG@it=\relax \let\PYG@bf=\relax%
    \let\PYG@ul=\relax \let\PYG@tc=\relax%
    \let\PYG@bc=\relax \let\PYG@ff=\relax}
\def\PYG@tok#1{\csname PYG@tok@#1\endcsname}
\def\PYG@toks#1+{\ifx\relax#1\empty\else%
    \PYG@tok{#1}\expandafter\PYG@toks\fi}
\def\PYG@do#1{\PYG@bc{\PYG@tc{\PYG@ul{%
    \PYG@it{\PYG@bf{\PYG@ff{#1}}}}}}}
\def\PYG#1#2{\PYG@reset\PYG@toks#1+\relax+\PYG@do{#2}}

\def\PYG@tok@gd{\def\PYG@tc##1{\textcolor[rgb]{0.63,0.00,0.00}{##1}}}
\def\PYG@tok@gu{\let\PYG@bf=\textbf\def\PYG@tc##1{\textcolor[rgb]{0.50,0.00,0.50}{##1}}}
\def\PYG@tok@gt{\def\PYG@tc##1{\textcolor[rgb]{0.00,0.25,0.82}{##1}}}
\def\PYG@tok@gs{\let\PYG@bf=\textbf}
\def\PYG@tok@gr{\def\PYG@tc##1{\textcolor[rgb]{1.00,0.00,0.00}{##1}}}
\def\PYG@tok@cm{\let\PYG@it=\textit\def\PYG@tc##1{\textcolor[rgb]{0.25,0.50,0.56}{##1}}}
\def\PYG@tok@vg{\def\PYG@tc##1{\textcolor[rgb]{0.73,0.38,0.84}{##1}}}
\def\PYG@tok@m{\def\PYG@tc##1{\textcolor[rgb]{0.13,0.50,0.31}{##1}}}
\def\PYG@tok@mh{\def\PYG@tc##1{\textcolor[rgb]{0.13,0.50,0.31}{##1}}}
\def\PYG@tok@cs{\def\PYG@tc##1{\textcolor[rgb]{0.25,0.50,0.56}{##1}}\def\PYG@bc##1{\colorbox[rgb]{1.00,0.94,0.94}{##1}}}
\def\PYG@tok@ge{\let\PYG@it=\textit}
\def\PYG@tok@vc{\def\PYG@tc##1{\textcolor[rgb]{0.73,0.38,0.84}{##1}}}
\def\PYG@tok@il{\def\PYG@tc##1{\textcolor[rgb]{0.13,0.50,0.31}{##1}}}
\def\PYG@tok@go{\def\PYG@tc##1{\textcolor[rgb]{0.19,0.19,0.19}{##1}}}
\def\PYG@tok@cp{\def\PYG@tc##1{\textcolor[rgb]{0.00,0.44,0.13}{##1}}}
\def\PYG@tok@gi{\def\PYG@tc##1{\textcolor[rgb]{0.00,0.63,0.00}{##1}}}
\def\PYG@tok@gh{\let\PYG@bf=\textbf\def\PYG@tc##1{\textcolor[rgb]{0.00,0.00,0.50}{##1}}}
\def\PYG@tok@ni{\let\PYG@bf=\textbf\def\PYG@tc##1{\textcolor[rgb]{0.84,0.33,0.22}{##1}}}
\def\PYG@tok@nl{\let\PYG@bf=\textbf\def\PYG@tc##1{\textcolor[rgb]{0.00,0.13,0.44}{##1}}}
\def\PYG@tok@nn{\let\PYG@bf=\textbf\def\PYG@tc##1{\textcolor[rgb]{0.05,0.52,0.71}{##1}}}
\def\PYG@tok@no{\def\PYG@tc##1{\textcolor[rgb]{0.38,0.68,0.84}{##1}}}
\def\PYG@tok@na{\def\PYG@tc##1{\textcolor[rgb]{0.25,0.44,0.63}{##1}}}
\def\PYG@tok@nb{\def\PYG@tc##1{\textcolor[rgb]{0.00,0.44,0.13}{##1}}}
\def\PYG@tok@nc{\let\PYG@bf=\textbf\def\PYG@tc##1{\textcolor[rgb]{0.05,0.52,0.71}{##1}}}
\def\PYG@tok@nd{\let\PYG@bf=\textbf\def\PYG@tc##1{\textcolor[rgb]{0.33,0.33,0.33}{##1}}}
\def\PYG@tok@ne{\def\PYG@tc##1{\textcolor[rgb]{0.00,0.44,0.13}{##1}}}
\def\PYG@tok@nf{\def\PYG@tc##1{\textcolor[rgb]{0.02,0.16,0.49}{##1}}}
\def\PYG@tok@si{\let\PYG@it=\textit\def\PYG@tc##1{\textcolor[rgb]{0.44,0.63,0.82}{##1}}}
\def\PYG@tok@s2{\def\PYG@tc##1{\textcolor[rgb]{0.25,0.44,0.63}{##1}}}
\def\PYG@tok@vi{\def\PYG@tc##1{\textcolor[rgb]{0.73,0.38,0.84}{##1}}}
\def\PYG@tok@nt{\let\PYG@bf=\textbf\def\PYG@tc##1{\textcolor[rgb]{0.02,0.16,0.45}{##1}}}
\def\PYG@tok@nv{\def\PYG@tc##1{\textcolor[rgb]{0.73,0.38,0.84}{##1}}}
\def\PYG@tok@s1{\def\PYG@tc##1{\textcolor[rgb]{0.25,0.44,0.63}{##1}}}
\def\PYG@tok@gp{\let\PYG@bf=\textbf\def\PYG@tc##1{\textcolor[rgb]{0.78,0.36,0.04}{##1}}}
\def\PYG@tok@sh{\def\PYG@tc##1{\textcolor[rgb]{0.25,0.44,0.63}{##1}}}
\def\PYG@tok@ow{\let\PYG@bf=\textbf\def\PYG@tc##1{\textcolor[rgb]{0.00,0.44,0.13}{##1}}}
\def\PYG@tok@sx{\def\PYG@tc##1{\textcolor[rgb]{0.78,0.36,0.04}{##1}}}
\def\PYG@tok@bp{\def\PYG@tc##1{\textcolor[rgb]{0.00,0.44,0.13}{##1}}}
\def\PYG@tok@c1{\let\PYG@it=\textit\def\PYG@tc##1{\textcolor[rgb]{0.25,0.50,0.56}{##1}}}
\def\PYG@tok@kc{\let\PYG@bf=\textbf\def\PYG@tc##1{\textcolor[rgb]{0.00,0.44,0.13}{##1}}}
\def\PYG@tok@c{\let\PYG@it=\textit\def\PYG@tc##1{\textcolor[rgb]{0.25,0.50,0.56}{##1}}}
\def\PYG@tok@mf{\def\PYG@tc##1{\textcolor[rgb]{0.13,0.50,0.31}{##1}}}
\def\PYG@tok@err{\def\PYG@bc##1{\fcolorbox[rgb]{1.00,0.00,0.00}{1,1,1}{##1}}}
\def\PYG@tok@kd{\let\PYG@bf=\textbf\def\PYG@tc##1{\textcolor[rgb]{0.00,0.44,0.13}{##1}}}
\def\PYG@tok@ss{\def\PYG@tc##1{\textcolor[rgb]{0.32,0.47,0.09}{##1}}}
\def\PYG@tok@sr{\def\PYG@tc##1{\textcolor[rgb]{0.14,0.33,0.53}{##1}}}
\def\PYG@tok@mo{\def\PYG@tc##1{\textcolor[rgb]{0.13,0.50,0.31}{##1}}}
\def\PYG@tok@mi{\def\PYG@tc##1{\textcolor[rgb]{0.13,0.50,0.31}{##1}}}
\def\PYG@tok@kn{\let\PYG@bf=\textbf\def\PYG@tc##1{\textcolor[rgb]{0.00,0.44,0.13}{##1}}}
\def\PYG@tok@o{\def\PYG@tc##1{\textcolor[rgb]{0.40,0.40,0.40}{##1}}}
\def\PYG@tok@kr{\let\PYG@bf=\textbf\def\PYG@tc##1{\textcolor[rgb]{0.00,0.44,0.13}{##1}}}
\def\PYG@tok@s{\def\PYG@tc##1{\textcolor[rgb]{0.25,0.44,0.63}{##1}}}
\def\PYG@tok@kp{\def\PYG@tc##1{\textcolor[rgb]{0.00,0.44,0.13}{##1}}}
\def\PYG@tok@w{\def\PYG@tc##1{\textcolor[rgb]{0.73,0.73,0.73}{##1}}}
\def\PYG@tok@kt{\def\PYG@tc##1{\textcolor[rgb]{0.56,0.13,0.00}{##1}}}
\def\PYG@tok@sc{\def\PYG@tc##1{\textcolor[rgb]{0.25,0.44,0.63}{##1}}}
\def\PYG@tok@sb{\def\PYG@tc##1{\textcolor[rgb]{0.25,0.44,0.63}{##1}}}
\def\PYG@tok@k{\let\PYG@bf=\textbf\def\PYG@tc##1{\textcolor[rgb]{0.00,0.44,0.13}{##1}}}
\def\PYG@tok@se{\let\PYG@bf=\textbf\def\PYG@tc##1{\textcolor[rgb]{0.25,0.44,0.63}{##1}}}
\def\PYG@tok@sd{\let\PYG@it=\textit\def\PYG@tc##1{\textcolor[rgb]{0.25,0.44,0.63}{##1}}}

\def\PYGZbs{\char`\\}
\def\PYGZus{\char`\_}
\def\PYGZob{\char`\{}
\def\PYGZcb{\char`\}}
\def\PYGZca{\char`\^}
\def\PYGZsh{\char`\#}
\def\PYGZpc{\char`\%}
\def\PYGZdl{\char`\$}
\def\PYGZti{\char`\~}
% for compatibility with earlier versions
\def\PYGZat{@}
\def\PYGZlb{[}
\def\PYGZrb{]}
\makeatother

\begin{document}

\maketitle
\tableofcontents
\phantomsection\label{index::doc}



\chapter{Introdução}
\label{index:bem-vindo-a-documentacao-do-projeto-pabx-ip}\label{index:introducao}
Esta documentação tem o objetivo de servir de base para auxiliar o entendimento do código-fonte do projeto PABX-IP, facilitando as futuras extensões de funcionalidades a serem desenvolvidas.


\chapter{Recomendações}
\label{index:recomendacoes}
Antes de começar, é importante que o leitor tenha um conhecimento geral de programação orientada a objetos, banco de dados e desenvolvimento web. É indispensável um conhecimento prévio no framework Django.


\chapter{Conteúdo}
\label{index:conteudo}

\section{Para começar}
\label{inicio::doc}\label{inicio:para-comecar}

\subsection{1. Introdução}
\label{inicio:introducao}
Nessa seção será mostrado as ferramentas que são pré-requisitos para desenvolvimento do projeto. Também mostraremos como instalá-las e configurá-las corretamente, para que o desenvolvedor já consiga ao menos rodar o PABX-IP em sua máquina local.


\subsection{2. Ambiente de Desenvolvimento}
\label{inicio:ambiente-de-desenvolvimento}\begin{itemize}
\item {} 
Sistema Operacional (recomendável): Ubuntu

\item {} 
Python 2.7 (a versão 3 é incompatível com outras bibliotecas)

\item {} 
Framework Django + Pinax

\item {} 
Banco de dados SQLite

\item {} 
Conexão com a internet

\end{itemize}


\subsection{3. Instalação do ambiente de desenvolvimento}
\label{inicio:instalacao-do-ambiente-de-desenvolvimento}
Esse tutorial foi feito no Ubuntu 10.04, usando o repositório git onde o código estava sendo versionado na época em que essa documentação foi feita. Caso o desenvolvedor já possua o código fonte, pule a parte 1.


\subsubsection{3.1. Instalação do git e clone do repositório:}
\label{inicio:instalacao-do-git-e-clone-do-repositorio}
\begin{Verbatim}[commandchars=\\\{\}]
\$ sudo apt-get install git-core

\$ git clone https://github.com/fabiothiroki/pabx\_ip.git
\end{Verbatim}


\subsubsection{3.2. Instalação e ativação do virtualenv:}
\label{inicio:instalacao-e-ativacao-do-virtualenv}
O virtualenv é uma ferramenta de criação de ambientes python isolados. Isso visa facilitar o deploy da aplicação.

Toda vez que o desenvolvedor quiser rodar o servidor local de desenvolvimento do django é necessário ativar esse ambiente, pois é nele que estarão instalados o Pinax e as bibliotecas python auxiliares.

Antes de mais nada é interessante que todas as bibliotecas do OS sejam atualizadas

\begin{Verbatim}[commandchars=\\\{\}]
\$ sudo apt-get update
\$ sudo apt-get dist-update
\end{Verbatim}

\begin{Verbatim}[commandchars=\\\{\}]
\$ sudo apt-get install python-pip

\$ sudo pip install virtualenv

\$ virtualenv pabx-env

\$ source pabx-env/bin/activate
\end{Verbatim}


\subsubsection{3.3. Instalação do Pinax e outros apps:}
\label{inicio:instalacao-do-pinax-e-outros-apps}
Pinax é uma plataforma baseada no Django que contém diversos apps pré-instalados.

Para mais informações consulte: \href{http://pinaxproject.com/}{http://pinaxproject.com/}

\begin{Verbatim}[commandchars=\\\{\}]
(pabx-env)\$ pip install Pinax
(pabx-env)\$ pip install django\_compressor
(pabx-env)\$ pip install django\_debug\_toolbar
(pabx-env)\$ pip install django\_staticfiles
(pabx-env)\$ pip install pinax\_theme\_bootstrap
\end{Verbatim}


\subsubsection{3.4. Instalação do Django:}
\label{inicio:instalacao-do-django}
Django é o framework web utilizado no projeto.

Para mais informações consulte: \href{http://djangoproject.com/}{http://djangoproject.com/}

\begin{Verbatim}[commandchars=\\\{\}]
(pabx-env)\$ pip install Django
\end{Verbatim}


\subsubsection{3.5. Instalação do Django South:}
\label{inicio:instalacao-do-django-south}
O South é utilizado para implementar o controle da estrutura e migração do banco de dados. Seus arquivos estão na pasta `south', no diretório raiz do projeto.

Para mais informações consulte: \href{http://south.aeracode.org/}{http://south.aeracode.org/}

\begin{Verbatim}[commandchars=\\\{\}]
(pabx-env)\$ pip install south
\end{Verbatim}


\subsubsection{3.6. Conclusão:}
\label{inicio:conclusao}
Por fim utilize o seguinte comando no diretório raiz do projeto para ligar o servidor de desenvolvimento:

\begin{Verbatim}[commandchars=\\\{\}]
\$ python manage.py runserver
\end{Verbatim}

A seguinte mensagem deverá ser retornada no terminal em caso de sucesso:

\begin{Verbatim}[commandchars=\\\{\}]
Validating models...

0 errors found
Django version 1.3.1, using settings 'pabx\_ip.settings'
Development server is running at http://127.0.0.1:8000/
Quit the server with CONTROL-C.
\end{Verbatim}

Entre com o endereço \href{http://127.0.0.1:8000/}{http://127.0.0.1:8000/} no seu navegador para acessar a interface web do projeto.

O banco de dados padrão do projeto vem com o usuário super-admin com login \emph{root} e senha \emph{1234}.


\section{Visão geral do código}
\label{codigo::doc}\label{codigo:visao-geral-do-codigo}
A linguagem escolhida (Python) para o desenvolvimento do projeto prioriza a legibilidade do código sobre a velocidade, bem como o framework Django. Assim, um desenvolvedor com uma certa experiência em Python não terá maiores problemas para entender o código, mesmo porque o mesmo se encontra num estágio inicial.

Ainda sim esse documento visa explicar o código mais detalhadamente e ao mesmo tempo dando alguma noção de Django.

Para que o desenvolvedor não fique confuso com a quantidade de arquivos do projeto, será listado aqui os apps desenvolvidos de fato (cada app corresponde a uma pasta na raiz do projeto):
\begin{itemize}
\item {} 
\textbf{Accounts}

\item {} 
\textbf{Groups}

\item {} 
\textbf{Skypelist}

\item {} 
\textbf{Smtp}

\end{itemize}

Além dos arquivos \emph{urls.py} e \emph{settings.py} nenhum dos outros arquivos foi codificado de fato anteriormente, mas foi gerado automaticamente pelo framework e suas ferramentas.

O código desenvolvido no projeto está seguindo a estrutura de apps do Django, assim como a documentação relativa a essa parte.

Cada App documentado contém seu arquivos numa pasta de mesmo nome na raiz principal do projeto. Assim, a estrutura da documentação de cada app passa a ser a seguinte:
\begin{itemize}
\item {} 
Formulários: relativo ao arquivo \emph{forms.py}

\item {} 
Modelos: relativo ao arquivo \emph{models.py}

\item {} 
Views: relativo ao arquivo \emph{views.py}

\item {} 
Decorators: relativo ao arquivo \emph{decorators.py} (opcional)

\item {} 
Templates: relativo aos arquivos dentro da pasta ``templates'' (opcional)

\end{itemize}

Eventualmente algum App possa precisar de arquivos templates adicionais que estarão contidos na pasta ``templates'' dentro da pasta do App. Além desses arquivos templates, os outros templates padrão estarão contidos na pasta ``templates'' dentro da pasta raiz.

O mesmo vale para os \emph{decorators}, nem todos os apps possuem o arquivo \emph{decorators.py} em seu diretório


\section{App Accounts}
\label{apps/accounts:accounts}\label{apps/accounts:app-accounts}\label{apps/accounts::doc}\begin{description}
\item[{Este app é responsável por:}] \leavevmode\begin{itemize}
\item {} 
Definir privilégios de administrador a usuários e limitar acesso a usuários comuns

\item {} 
Cadastro, edição, listagem e remoção de usuários

\item {} 
Autenticação e recuperação de senha

\end{itemize}

\end{description}


\subsection{Modelos}
\label{apps/accounts:module-accounts.models}\label{apps/accounts:modelos}\index{accounts.models (módulo)}
No arquivo models.py estão as classes definidas para armazenar informações extras do usuário. Podemos observar que a classe UserProfile tem uma chave estrangeira na classe User, que é a classe padrão para usuários do Django. Desta maneira, podemos extender os atributos da classe User sem alterar a estrutura de usuário do Django, bastando apenas fazer essa pequena extensão.

Os atributos definidos na classe UserProfile estão detalhados a seguir:
\index{UserProfile (classe em accounts.models)}

\begin{fulllineitems}
\phantomsection\label{apps/accounts:accounts.models.UserProfile}\pysigline{\strong{class }\code{accounts.models.}\bfcode{UserProfile}}~\begin{itemize}
\item {} 
\textbf{profile}: Chave estrangeira que associa um UserProfile a um Usuário. Pode haver apenas um UserProfile por User.

\item {} 
\textbf{ramal}: Ramal associado a um usuário. Esse número é utilizado para fazer ligações.

\item {} 
\textbf{passw}: É a cópia da senha do usuário salva encriptadamente. É utilizada para recuperação de senha.

\item {} 
\textbf{admin}: Indica se o usuário possui o privilégio de administrador.

\item {} 
\textbf{group}: Indica o grupo o qual o usuário pertence.

\end{itemize}

\end{fulllineitems}



\subsection{Formulários}
\label{apps/accounts:module-accounts.forms}\label{apps/accounts:formularios}\index{accounts.forms (módulo)}\index{UserForm (classe em accounts.forms)}

\begin{fulllineitems}
\phantomsection\label{apps/accounts:accounts.forms.UserForm}\pysigline{\strong{class }\code{accounts.forms.}\bfcode{UserForm}}
Formulário utilizado pelo administrador para editar ou criar usuários. Além dos campos padrões relativos a classe User e a classe UserProfile existe um \emph{hidden input} que contém o id do usuário em caso de edição, para que possamos validar as informações na hora de salvar no banco de dados.

O método clean é sobreescrito para validarmos a confirmação de senha e no caso de edição, temos que permitir o salvamento de um ramal ou email já existente.

\end{fulllineitems}

\index{OnlyUserForm (classe em accounts.forms)}

\begin{fulllineitems}
\phantomsection\label{apps/accounts:accounts.forms.OnlyUserForm}\pysigline{\strong{class }\code{accounts.forms.}\bfcode{OnlyUserForm}}
Formulário utilizado para que um usuário sem privilégios de administrador possa mudar seu email ou senha.

\end{fulllineitems}

\index{PassResetForm (classe em accounts.forms)}

\begin{fulllineitems}
\phantomsection\label{apps/accounts:accounts.forms.PassResetForm}\pysigline{\strong{class }\code{accounts.forms.}\bfcode{PassResetForm}}
Formulário utilizado para que um usuário possa receber sua senha esquecida no seu email. Apenas emails cadastrados são aceitos.

\end{fulllineitems}



\subsection{Views}
\label{apps/accounts:module-accounts.views}\label{apps/accounts:views}\index{accounts.views (módulo)}\index{login() (no módulo accounts.views)}

\begin{fulllineitems}
\phantomsection\label{apps/accounts:accounts.views.login}\pysiglinewithargsret{\code{accounts.views.}\bfcode{login}}{\emph{request}}{}
View que inicialmente mostra a tela de login para o usuário, caso o usuário entre na página inicial do projeto ou tente acessar alguma outra página através da url sem estar logado. O usuário ao submeter o formulário de login através de um método POST, fará com que a view tente autenticar esse usuário, e em caso de sucesso, guardará na sessão se o usuário é admininistrador, seu username e seu id e o redirecionará para a página principal. Em caso de falha, a view retorna uma mensagem de erro para o template de login.

\end{fulllineitems}

\index{logout() (no módulo accounts.views)}

\begin{fulllineitems}
\phantomsection\label{apps/accounts:accounts.views.logout}\pysiglinewithargsret{\code{accounts.views.}\bfcode{logout}}{\emph{request}}{}
Faz o logout do usuário logado e o redireciona para a página de login.

\end{fulllineitems}

\index{settings() (no módulo accounts.views)}

\begin{fulllineitems}
\phantomsection\label{apps/accounts:accounts.views.settings}\pysiglinewithargsret{\code{accounts.views.}\bfcode{settings}}{\emph{request}}{}
É a tela que lista todos os usuários do pabx-ip e permite que o administrador escolha qual usuário editar ou remover através de uma interface. Também possui um botão para a tela de cadastro de usuários. É importante lembrar que somente o usuário `root' pode editar ou remover outros administradores.

\end{fulllineitems}

\index{create() (no módulo accounts.views)}

\begin{fulllineitems}
\phantomsection\label{apps/accounts:accounts.views.create}\pysiglinewithargsret{\code{accounts.views.}\bfcode{create}}{\emph{request}}{}
View que usa o UserForm para cadastrar um novo usuário no sistema. Somente acessível para administradores.

\end{fulllineitems}

\index{edit() (no módulo accounts.views)}

\begin{fulllineitems}
\phantomsection\label{apps/accounts:accounts.views.edit}\pysiglinewithargsret{\code{accounts.views.}\bfcode{edit}}{\emph{request}, \emph{offset}}{}
View que usa o UserForm para editar um usuário pré-cadastrado no sistema. Através do offset passado pela url a view sabe o id do usuário que se deseja modificar. Somente acessível para administradores.

\end{fulllineitems}

\index{delete() (no módulo accounts.views)}

\begin{fulllineitems}
\phantomsection\label{apps/accounts:accounts.views.delete}\pysiglinewithargsret{\code{accounts.views.}\bfcode{delete}}{\emph{request}}{}
View usada para remover um usuário do sistema. Através do offset passado pela url a view sabe o id do usuário que se deseja deletar. Primeiramente exibe uma tela de confirmação, e em seguida caso haja confirmação por parte do administrador, a classe User e sua respectiva classe UserProfile são removidas do banco de dados.

\end{fulllineitems}

\index{edit\_self() (no módulo accounts.views)}

\begin{fulllineitems}
\phantomsection\label{apps/accounts:accounts.views.edit_self}\pysiglinewithargsret{\code{accounts.views.}\bfcode{edit\_self}}{\emph{request}}{}
View que o usa o OnlyUserForm para que um usuário comum logado para editar seu email ou senha.

\end{fulllineitems}

\index{save\_or\_update() (no módulo accounts.views)}

\begin{fulllineitems}
\phantomsection\label{apps/accounts:accounts.views.save_or_update}\pysiglinewithargsret{\code{accounts.views.}\bfcode{save\_or\_update}}{\emph{form}, \emph{user=None}, \emph{profile=None}}{}
Método que faz a associação entre um form e os objetos User e UserProfile. Caso os parâmetros user e profile não sejam vazios, a função interpreta como edição de usuário, e terá que fazer a busca dele no banco.

\end{fulllineitems}

\index{password\_reset() (no módulo accounts.views)}

\begin{fulllineitems}
\phantomsection\label{apps/accounts:accounts.views.password_reset}\pysiglinewithargsret{\code{accounts.views.}\bfcode{password\_reset}}{\emph{request}}{}
Retorna inicialmente o template do formulário PassResetForm onde o usuário deverá digitar um email cadastrado válido. Após a submissão do formulário, o sistema checa se existe um servidor smtp pré-cadastrado pelo administrador para envio de emails. Em caso positivo, o email é enviado, e retorna o template indicando uma mensagem de sucesso.

\end{fulllineitems}

\index{encrypt() (no módulo accounts.views)}

\begin{fulllineitems}
\phantomsection\label{apps/accounts:accounts.views.encrypt}\pysiglinewithargsret{\code{accounts.views.}\bfcode{encrypt}}{\emph{plaintext}}{}
Função que retorna a variável \emph{plaintext} encriptada. Usada para salvar a senha encriptada dos usuários no banco.

\end{fulllineitems}

\index{unencrypt() (no módulo accounts.views)}

\begin{fulllineitems}
\phantomsection\label{apps/accounts:accounts.views.unencrypt}\pysiglinewithargsret{\code{accounts.views.}\bfcode{unencrypt}}{\emph{encrypted\_password}}{}
Função que retorna a variável \emph{encrypted\_password} desencriptada. Usada para recuperar a senha dos usuários em caso de esquecimento.

\end{fulllineitems}



\subsection{Templates}
\label{apps/accounts:templates}
Aqui serão listados os templates específicos utilizados por esse App, contidos na pasta ``accounts/templates/''
\begin{itemize}
\item {} 
password\_reset\_form.html: utilizado para renderizar o formulário para recuperação de senha.

\item {} 
password\_reset\_success.html: utilizado para mostrar a mensagem de sucesso na recuperação de senha.

\end{itemize}


\subsection{Decorators}
\label{apps/accounts:module-accounts.decorators}\label{apps/accounts:decorators}\index{accounts.decorators (módulo)}\index{is\_admin() (no módulo accounts.decorators)}

\begin{fulllineitems}
\phantomsection\label{apps/accounts:accounts.decorators.is_admin}\pysiglinewithargsret{\code{accounts.decorators.}\bfcode{is\_admin}}{\emph{function}}{}
Checa se o User logado possui o atributo admin na respectiva classe UserProfile. Usado para limitar o acesso a certas Views que apenas administradores podem acessar.

\end{fulllineitems}



\section{App Groups}
\label{apps/groups:app-groups}\label{apps/groups::doc}\label{apps/groups:groups}\begin{description}
\item[{Este app é responsável por:}] \leavevmode\begin{itemize}
\item {} 
Criar, editar e excluir grupos

\item {} 
Associar permissões de usuários a um ou mais grupos

\end{itemize}

\end{description}

Todas as funcionalidades de grupos é apenas acessível para administradores.

\begin{notice}{warning}{Aviso:}
Esse App não foi finalizado ainda, portanto não atende ainda a todos os requisitos atuais do projeto. Porém no estágio em que ele está desenvolvido é possível rodar o App com as funcionalidades atuais sem problemas.
\end{notice}


\subsection{Modelos}
\label{apps/groups:modelos}\label{apps/groups:module-groups.models}\index{groups.models (módulo)}
Existe uma única classe \emph{Group} definida aqui, que corresponde obviamente a um grupo. Um uso prático para essa classe é poder separar grupos de usuários por departamento, por exemplo: ``grupo recepção'' e ``grupo gerência''.

A associação entre grupos e usuários está definida no model \emph{UserProfile} no qual o atributo \emph{group} define o grupo do usuário. Portanto, um grupo possui vários usuários, mas um usuário só pode se associar a um grupo.

Os atributos da classe grupo são:
\index{Group (classe em groups.models)}

\begin{fulllineitems}
\phantomsection\label{apps/groups:groups.models.Group}\pysigline{\strong{class }\code{groups.models.}\bfcode{Group}}~\begin{itemize}
\item {} 
\textbf{name}: nome do grupo, usado apenas para identificá-lo.

\item {} 
\textbf{can\_call\_ramal}: permissão dada como padrão para todos os grupos, mantemos aqui apenas para fins de visualização.

\item {} 
\textbf{can\_call\_emergency}: permissão dada como padrão para todos os grupos, mantemos aqui apenas para fins de visualização.

\item {} 
\textbf{can\_call\_fix}: indica se o grupo pode fazer ligações para fixo local.

\item {} 
\textbf{can\_call\_mobile}: indica se o grupo pode fazer ligações para celular local.

\item {} 
\textbf{can\_call\_ddd}: indica se grupo pode fazer ligações DDD.

\item {} 
\textbf{can\_call\_ddi}: indica se grupo pode fazer ligações DDI.

\item {} 
\textbf{can\_call\_0800}: indica se grupo pode fazer ligações 0800.

\item {} 
\textbf{can\_call\_0300}: indica se grupo pode fazer ligações 0300.

\end{itemize}

A função \emph{unicode} serve para retornar o nome do grupo no caso de imprimirmos algum objeto Group.

\end{fulllineitems}



\subsection{Formulários}
\label{apps/groups:module-groups.forms}\label{apps/groups:formularios}\index{groups.forms (módulo)}\index{GroupForm (classe em groups.forms)}

\begin{fulllineitems}
\phantomsection\label{apps/groups:groups.forms.GroupForm}\pysigline{\strong{class }\code{groups.forms.}\bfcode{GroupForm}}
\emph{ModelForm} que utiliza como modelo a classe \emph{Group}. A partir dele é possível criar ou editar grupos, e a única instância feita por essa classe é desabilitar o checkbox de permissão para ligar para ramal e para emergência, pois estas permissões são dadas como padrão para qualquer grupo.

\end{fulllineitems}



\subsection{Views}
\label{apps/groups:module-groups.views}\label{apps/groups:views}\index{groups.views (módulo)}\index{index() (no módulo groups.views)}

\begin{fulllineitems}
\phantomsection\label{apps/groups:groups.views.index}\pysiglinewithargsret{\code{groups.views.}\bfcode{index}}{\emph{request}}{}
Função que retorna a lista com os grupos cadastrados, utilizando o template \emph{crud}. Esse template possui links para edição, remoção e criação de grupos.

\end{fulllineitems}

\index{create() (no módulo groups.views)}

\begin{fulllineitems}
\phantomsection\label{apps/groups:groups.views.create}\pysiglinewithargsret{\code{groups.views.}\bfcode{create}}{\emph{request}}{}
View que utiliza o \emph{GroupForm} para criar novos grupos.

\end{fulllineitems}



\section{App Skypelist}
\label{apps/skypelist:app-skypelist}\label{apps/skypelist::doc}\label{apps/skypelist:skypelist}\begin{description}
\item[{Esse app é responsável por:}] \leavevmode\begin{itemize}
\item {} 
Criar, editar e remover contatos skype

\end{itemize}

\end{description}

Um contato Skype é apenas uma associação entre um nome de usuário Skype pré-cadastrado e um ramal do pabx, sendo que este ramal deve ser único, independente de ser um ramal normal ou ``skype''. A idéia por trás disso está em possibilitar o usuário a fazer ligações skype de um telefone normal, conectado ao PABX-IP.

Todas essas funcionalidades desse App está apenas acessível para Administradores.


\subsection{Modelos}
\label{apps/skypelist:module-skypelist.skypeuser}\label{apps/skypelist:modelos}\index{skypelist.skypeuser (módulo)}
Existe uma uńica classe \emph{skypeuser} definida aqui, que corresponde a associação entre nome de usuário skype e ramal.

Os atributos da classe \emph{skypeuser} são:
\index{skypeuser (classe em skypelist.skypeuser)}

\begin{fulllineitems}
\phantomsection\label{apps/skypelist:skypelist.skypeuser.skypeuser}\pysigline{\strong{class }\code{skypelist.skypeuser.}\bfcode{skypeuser}}~\begin{itemize}
\item {} 
\textbf{ramal}: número do ramal.

\item {} 
\textbf{username}: nome de usuário Skype.

\end{itemize}

\end{fulllineitems}



\subsection{Formulários}
\label{apps/skypelist:module-skypelist.forms}\label{apps/skypelist:formularios}\index{skypelist.forms (módulo)}\index{skypeform (classe em skypelist.forms)}

\begin{fulllineitems}
\phantomsection\label{apps/skypelist:skypelist.forms.skypeform}\pysigline{\strong{class }\code{skypelist.forms.}\bfcode{skypeform}}
\emph{ModelForm} que utiliza como modelo a classe \emph{skypeuser}. Acrescenta um campo \emph{hidden input} para tratar o caso de edição do usuário, para evitar a validação de campos com mesmos valores.

\end{fulllineitems}



\subsection{Views}
\label{apps/skypelist:views}\label{apps/skypelist:module-skypelist.views}\index{skypelist.views (módulo)}\index{index() (no módulo skypelist.views)}

\begin{fulllineitems}
\phantomsection\label{apps/skypelist:skypelist.views.index}\pysiglinewithargsret{\code{skypelist.views.}\bfcode{index}}{\emph{request}}{}
Função que retorna a lista com os usuários skype cadastrados, utilizando o template \emph{crud}. Esse template possui links para edição, remoção e criação.

\end{fulllineitems}

\index{create() (no módulo skypelist.views)}

\begin{fulllineitems}
\phantomsection\label{apps/skypelist:skypelist.views.create}\pysiglinewithargsret{\code{skypelist.views.}\bfcode{create}}{\emph{request}}{}
View que utiliza o \emph{skypeform} para criar novos usuários skype.

\end{fulllineitems}

\index{edit() (no módulo skypelist.views)}

\begin{fulllineitems}
\phantomsection\label{apps/skypelist:skypelist.views.edit}\pysiglinewithargsret{\code{skypelist.views.}\bfcode{edit}}{\emph{request}, \emph{offset}}{}
View que utiliza o \emph{skypeform} para editar usuários skype pré-cadastrados. Através do offset passado pela url a view sabe o id do usuário skype que se deseja modificar.

\end{fulllineitems}



\begin{fulllineitems}
\pysigline{\bfcode{delete(request,offset):}}
View usada para remover um usuário skype do sistema. Através do offset passado pela url a view sabe o id do usuário skype que se deseja deletar.Primeiramente exibe uma tela de confirmação, e em seguida caso haja confirmação por parte do administrador, a classe \emph{skypuser} é removida do banco de dados.

\end{fulllineitems}



\section{App Smtp}
\label{apps/smtp:smtp}\label{apps/smtp:app-smtp}\label{apps/smtp::doc}
Esse App é responsável pelo cadastro de um servidor smtp que será usado pelo PABX-IP para enviar emails diversos. A partir de um servidor smtp configurado para o PABX-IP é possível, por exemplo, mandar emails a partir de uma conta cadastrada no gmail.

Apenas um único servidor smtp é permitido, sendo que somente o Administrador pode alterar suas configurações.


\subsection{Modelos}
\label{apps/smtp:module-smtp.server}\label{apps/smtp:modelos}\index{smtp.server (módulo)}
Existe uma única classe \emph{server} definida aqui, que corresponde aos dados do servidor smtp.

Os atributos da classe \emph{server} são:
\index{server (classe em smtp.server)}

\begin{fulllineitems}
\phantomsection\label{apps/smtp:smtp.server.server}\pysigline{\strong{class }\code{smtp.server.}\bfcode{server}}~\begin{itemize}
\item {} 
\textbf{url}: url para o servidor smtp.

\item {} 
\textbf{port}: porta do servidor smtp.

\item {} 
\textbf{username}: nome de usuário do servidor smtp.

\item {} 
\textbf{password}: senha do servidor smtp.

\end{itemize}

\end{fulllineitems}



\subsection{Formulários}
\label{apps/smtp:module-smtp.forms}\label{apps/smtp:formularios}\index{smtp.forms (módulo)}\index{smtpform (classe em smtp.forms)}

\begin{fulllineitems}
\phantomsection\label{apps/smtp:smtp.forms.smtpform}\pysigline{\strong{class }\code{smtp.forms.}\bfcode{smtpform}}
Nesse \emph{ModelForm} cujo modelo é a classe \emph{server}, acrescentamos o campo de confirmação de senha e a validação da mesma. Esse formulário é utilizado tanto para cadastro como edição do servidor smtp.

\end{fulllineitems}



\subsection{Views}
\label{apps/smtp:module-smtp.views}\label{apps/smtp:views}\index{smtp.views (módulo)}\index{index() (no módulo smtp.views)}

\begin{fulllineitems}
\phantomsection\label{apps/smtp:smtp.views.index}\pysiglinewithargsret{\code{smtp.views.}\bfcode{index}}{\emph{request}}{}
Função que retorna os dados do servidor smtp e imprime num template \emph{crud}. Se o servidor não estiver configurado ainda, uma mensagem será exibida. Essa tela ainda possui links para cadastrar ou editar um servidor smtp.

\end{fulllineitems}

\index{configure() (no módulo smtp.views)}

\begin{fulllineitems}
\phantomsection\label{apps/smtp:smtp.views.configure}\pysiglinewithargsret{\code{smtp.views.}\bfcode{configure}}{\emph{request}}{}
Função que permite o cadastro ou a edição do servidor smtp, utilizando o \emph{smtpform}.

\end{fulllineitems}



\section{Bibliotecas externas}
\label{bibliotecas:bibliotecas-externas}\label{bibliotecas::doc}

\subsection{1. Introdução}
\label{bibliotecas:introducao}
Nessa seção listaremos as bibliotecas auxiliares ao projeto, que já vem previamente instaladas e configuradas. O desenvolvedor deve ter conhecimento delas caso precise utilizá-las futuramente.

Cada biblioteca utilizada possui também uma documentação própria, que pode elucidar dúvidas mais específicas.


\subsection{2. Twitter Bootstrap}
\label{bibliotecas:twitter-bootstrap}
\href{http://twitter.github.com/bootstrap/}{http://twitter.github.com/bootstrap/}

Utilizado como base da identidade visual do projeto. Embora o Pinax já venha com um template padrão usando o Twitter Bootstrap, foi preferível começar um novo template para fins de customização.


\subsection{3. jQuery}
\label{bibliotecas:jquery}
\href{http://jquery.com/}{http://jquery.com/}

É o framework Javascript utilizado no projeto.


\section{Futuras implementações}
\label{futuro::doc}\label{futuro:futuras-implementacoes}

\renewcommand{\indexname}{Índice de Módulos do Python}
\begin{theindex}
\def\bigletter#1{{\Large\sffamily#1}\nopagebreak\vspace{1mm}}
\bigletter{a}
\item {\texttt{accounts.decorators}}, \pageref{apps/accounts:module-accounts.decorators}
\item {\texttt{accounts.forms}}, \pageref{apps/accounts:module-accounts.forms}
\item {\texttt{accounts.models}}, \pageref{apps/accounts:module-accounts.models}
\item {\texttt{accounts.views}}, \pageref{apps/accounts:module-accounts.views}
\indexspace
\bigletter{g}
\item {\texttt{groups.forms}}, \pageref{apps/groups:module-groups.forms}
\item {\texttt{groups.models}}, \pageref{apps/groups:module-groups.models}
\item {\texttt{groups.views}}, \pageref{apps/groups:module-groups.views}
\indexspace
\bigletter{s}
\item {\texttt{skypelist.forms}}, \pageref{apps/skypelist:module-skypelist.forms}
\item {\texttt{skypelist.skypeuser}}, \pageref{apps/skypelist:module-skypelist.skypeuser}
\item {\texttt{skypelist.views}}, \pageref{apps/skypelist:module-skypelist.views}
\item {\texttt{smtp.forms}}, \pageref{apps/smtp:module-smtp.forms}
\item {\texttt{smtp.server}}, \pageref{apps/smtp:module-smtp.server}
\item {\texttt{smtp.views}}, \pageref{apps/smtp:module-smtp.views}
\end{theindex}

\renewcommand{\indexname}{Índice}
\printindex
\end{document}
